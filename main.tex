\documentclass[11pt]{article}
\usepackage{amsmath,amssymb, amsthm, marvosym, permute, extsizes}
\usepackage{siunitx, graphicx, float, enumitem, adjustbox, hyperref, bm}
\usepackage{microtype, dsfont, subcaption}
\usepackage[normalem]{ulem}
\usepackage[T1]{fontenc}
\usepackage[utf8]{inputenc}
\usepackage{lmodern}
\usepackage[T1]{fontenc}
\usepackage[a4paper,margin=2.5cm]{geometry}
\usepackage[icelandic]{babel}
\title{Heimadæmi 1\\ \vspace{0.4cm} \large Frumeinda- og ljósfræði}
\author{Emil Gauti Friðriksson}
\begin{document}
\maketitle
\section*{Dæmi 1}

\subsection*{Haken \& Wolf dæmi 11.1}
The energy levels of the valence electrons of an alkali atom ar egiven, to a good approximation, by the expression
$$E_n  =- Rhc\cdot\frac{1}{[n-\Delta(n,l)]^2}$$
\noindent Here $\Delta (n,l)$ is the quantum defect(which depends on the values of $n$ and $l$ of the valence electron in question). For lithium and sodium, $\Delta(n,l)$ have been measured:\\

\begin{tabular}{ l l l l }
\hline
\quad 		& $s$ & $p$ & $d$\\
\hline
Li($z=3$)  	& 0.40 & 0.04 & 0.00\\
Na($z=11$) 	& 1.37& 0.88 & 0.01\\
\hline
\end{tabular}\\

\noindent Calculate the energy of the ground state and the first two excited states of the valence electron in lithium and sodium.\\

\subsection*{Svar}
Við munum nýta okkur að $Rhc=13.605 eV$\\
Athugum Li frumeind, Li hefur 3 rafeindir og í grunnástandi situr ysta rafeindin í 2s svigrúminu. Orka grunnástandsins er því eftirfarandi
\begin{align*}
E_0  	&= -Rhc\cdot\frac{1}{[2-0.4]^2}\\
		&= -Rhc\cdot \frac{25}{64}\\
        &= \underline{\underline{-5.31 eV}}
\end{align*}
Athugum nú fyrsta örvaða ástand, þá hefur ysta rafeindin stokkið upp á næsta svigrum eða 2p svigrúmið

\begin{align*}
E_1 &= -Rhc\cdot\frac{1}{[2-0.04]^2}\\
	&= \underline{\underline{-3.54 eV}}
\end{align*}

\noindent Athugum nú annað örvaða ástandið, þá hefur rafeindin stokkið upp í 3s svigrúmið og því er orkan eftirfarandi

\begin{align*}
E_2 &= -Rhc\cdot \frac{1}{[3-0.4]^2}\\
	&= \underline{\underline{-2.01 eV}}
\end{align*}

\noindent Lítum nú á Na frumeindina. Na hefur 14 rafeindir og í grunnástandi situr ysta rafeindin í $3s^1$ svigrúminu, því er orka grunnástandsins

\begin{align*}
E_0 &= -Rhc \cdot \frac{1}{[3-1.37]^2}\\
	&= \underline{\underline{-5.12 eV}}
\end{align*}

\noindent Athugum nú fyrsta örvaða ástand, þá hefur rafeindin stokkið upp í 3p svigrúmið

\begin{align*}
E_1 &= -Rhc \cdot \frac{1}{[3-0.88]^2}\\
	&= \underline{\underline{-3.03 eV}}
\end{align*}

\noindent Athugum nú annað örvaða ástand, þá hefur rafeindin stokkið upp í 4s svigrúmið

\begin{align*}
E_2 &= Rhc \cdot \frac{1}{[4-1.37]^2}\\
	&= \underline{\underline{-1.97 eV}}
\end{align*}

\section*{Dæmi 2}
\subsection*{Haken \& Wolf dæmi 11.2}
The ionisation energy of the Li atom is $5.3913$ eV, and the resonance line $(2s\leftrightarrow 2p)$ is observed at $6710 Å$. Lithium vapour is selectively excited so that only the 3p level is occupied. Which spectral lines are emitted by this vapour, and what are their wavelengths?

\subsection*{Svar}
Þegar ysta rafeindin er stödd í 3p svigrúminu getur hún 'dottið' niður á 2 önnur svigrúm- 3s og 2s svigrúmin. Úr 3s  svigrúminu getur rafeindin 'dottið' enn aftur. Orka 3p ástandsins er fundið líkt og í dæminu hér á undan og fæst gildið $-1.55eV$. Skrifum nú upp alla möguleikana, orkumismun ástandanna og bylgjulengd ljóss sem fylgir hverjum orkumismun:
\begin{align*}
&3p\rightarrow 3s \Rightarrow \Delta E = 0.46 eV \Rightarrow \lambda = 2697nm\\
&3p\rightarrow 2s \Rightarrow \Delta E = 3.76 eV \Rightarrow \lambda = 330nm\\
\quad\\
&3s\rightarrow 2p \Rightarrow \Delta E = 1.53 eV \Rightarrow \lambda = 810nm\\
\quad\\
&2p\rightarrow 2s \Rightarrow \Delta E = 1.77 eV \Rightarrow \lambda = 701nm\\
\end{align*}





\section*{Dæmi 3}
\subsection*{Haken \& Wolf dæmi 11.3}
Explain the symbols for the $3^2D\rightarrow 3^2P$ transition in sodium. How many lines can be expected in the spectrum?
\subsection*{Svar}
Þristurinn í $3^2D\rightarrow 3^2P$ merkir að aðalskammtatalan($n$) er jöfn þremur, $D$ og $P$ segja til um í hvaða svigrúmi rafeindirnar eru, $D$ í d-svigrúmi og $P$ í p-svigrúmi, ($n ^{2s+1}\Lambda_J$).\\
Fyrir vinstri hlið gildir að $l=2$ einnig gildir $s=\frac 12$ sem gefur að $j=\frac
32 , \frac 52$\\
Fyrir hægri hlið gildir að $l=1$, $s=\frac 12$ $\Rightarrow j = \frac 12, \frac 32$\\
Leyfileg gildi fyrir $\Delta l, \Delta s$ og $\Delta j$ er: $\Delta l = \pm 1, \Delta s = 0$ og $\Delta j = 0,\pm 1$ þess vegna fáum við 3 línur: ein þegar $3^2D_\frac{3}{2} \rightarrow 3^2P_\frac{3}{2}$, ein þegar $3^2D_\frac{3}{2} \rightarrow 3^2P_\frac{1}{2}$ og loks $3^2D_\frac{5}{2} \rightarrow 3^2P_\frac{3}{2}$

\section*{Dæmi 4}
\subsection*{Haken \& Wolf dæmi 11.3 fyrir $3^2D\rightarrow 4^2F$ færslu natríns}

\subsection*{Svar}
Fyrri hluti dæmisins er fullkomlega sambærilegur fyrri hluta Dæmi 3, einbeitum okkur því að finna út fjölda lína.\\
Fyrir vinstri hlið gildir að $l=2 \Rightarrow j = \frac{3}{2}, \frac 52$\\
Fyrir hægri hlið gildir að $l=3 \Rightarrow j = \frac 72, \frac 52$\\
Eins og sagt var í dæmi 3 þá gildir ávallt að $\Delta l = \pm 1, \Delta s = 0$ og $\Delta j = 0,\pm 1$ því fást 3 línur:\\
ein þegar $3^2D_\frac{5}{2} \rightarrow 3^2F_\frac{5}{2}$, ein þegar $3^2D_\frac{3}{2} \rightarrow 3^2F_\frac{5}{2}$ og loks $3^2D_\frac{5}{2} \rightarrow 3^2F_\frac{7}{2}$


\end{document}
